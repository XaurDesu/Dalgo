\documentclass[a4paper]{article} 
\addtolength{\hoffset}{-2.25cm}
\addtolength{\textwidth}{4.5cm}
\addtolength{\voffset}{-3.25cm}
\addtolength{\textheight}{5cm}
\setlength{\parskip}{0pt}
\setlength{\parindent}{0in}

%----------------------------------------------------------------------------------------
%	PACKAGES AND OTHER DOCUMENT CONFIGURATIONS
%----------------------------------------------------------------------------------------

\usepackage{blindtext} % Package to generate dummy text
\usepackage{charter} % Use the Charter font
\usepackage[utf8]{inputenc} % Use UTF-8 encoding
\usepackage{microtype} % Slightly tweak font spacing for aesthetics
\usepackage[english, ngerman]{babel} % Language hyphenation and typographical rules
\usepackage{amsthm, amsmath, amssymb} % Mathematical typesetting
\usepackage{float} % Improved interface for floating objects
\usepackage[final, colorlinks = true, 
            linkcolor = black, 
            citecolor = black]{hyperref} % For hyperlinks in the PDF
\usepackage{graphicx, multicol} % Enhanced support for graphics
\usepackage{xcolor} % Driver-independent color extensions
\usepackage{marvosym, wasysym} % More symbols
\usepackage{rotating} % Rotation tools
\usepackage{censor} % Facilities for controlling restricted text
\usepackage{listings, style/lstlisting} % Environment for non-formatted code, !uses style file!
\usepackage{pseudocode} % Environment for specifying algorithms in a natural way
\usepackage{style/avm} % Environment for f-structures, !uses style file!
\usepackage{booktabs} % Enhances quality of tables
\usepackage{tikz-qtree} % Easy tree drawing tool
\tikzset{every tree node/.style={align=center,anchor=north},
         level distance=2cm} % Configuration for q-trees
\usepackage{style/btree} % Configuration for b-trees and b+-trees, !uses style file!
\usepackage[backend=biber,style=numeric,
            sorting=nyt]{biblatex} % Complete reimplementation of bibliographic facilities
\addbibresource{ecl.bib}
\usepackage{csquotes} % Context sensitive quotation facilities
\usepackage[yyyymmdd]{datetime} % Uses YEAR-MONTH-DAY format for dates
\renewcommand{\dateseparator}{-} % Sets dateseparator to '-'
\usepackage{fancyhdr} % Headers and footers
\pagestyle{fancy} % All pages have headers and footers
\fancyhead{}\renewcommand{\headrulewidth}{0pt} % Blank out the default header
\fancyfoot[L]{} % Custom footer text
\fancyfoot[C]{} % Custom footer text
\fancyfoot[R]{\thepage} % Custom footer text
\newcommand{\note}[1]{\marginpar{\scriptsize \textcolor{red}{#1}}} % Enables comments in red on margin

%----------------------------------------------------------------------------------------

\begin{document}

%-------------------------------
%	TITLE SECTION
%-------------------------------

\fancyhead[C]{}
\hrule \medskip % Upper rule
\begin{minipage}{0.295\textwidth} 
\raggedright
\footnotesize
Jaime Andres Torres Bermejo \hfill\\   
202014866\hfill\\
andrestbermejoj@gmail.com
\end{minipage}
\begin{minipage}{0.4\textwidth} 
\centering 
\large 
Taller 5\\ 
\normalsize 
Diseño y Análisis de Algoritmos\\ 
\end{minipage}
\begin{minipage}{0.295\textwidth} 
\raggedleft
\today\hfill\\
\end{minipage}
\medskip\hrule 
\bigskip

%-------------------------------
%	CONTENIDO
%-------------------------------
\section{Explique y justifique que pasa cuando el algoritmo de Dijkstra recibe un grafo con
un ciclo negativo.}

El algoritmo de Dijkstra nos permite buscar el camino de menor longitud entre 2 nodos de un grafo,
y un ciclo negativo es un ciclo en el cuál los valores de todos los vertices del ciclo
den una suma total negativa. El algoritmo de Dijkstra es un algoritmo sin uso de 
recursividad o Programación dinámica en la mayoría de sus implementaciones, en su lugar utilizando
una estrategia Greedy para resolver el problema, esta estrategia implica que el algoritmo
seleccionará la solución localmente óptima a la hora de ejecutarse. Al introducirse un ciclo negativo
en un algoritmo de este tipo, que inherentemente esta solamente preocupado por entregar una solución
inmediatamente menor, puede causar situaciones de este tipo:
    
\begin{verbatim}
      A
     / \
    /   \
   /     \
  5       2
 /         \
B--(-800)--  C

V={A,B,C} ; E = {(A,C,2), (A,B,5), (B,C,-800)}
\end{verbatim}

En este caso tenemos un ciclo negativo y podemos asumir que esta conectado a mas partes
del grafo,  Si quisieramos encontrar un camino de un nodo a otro, sería 
posible que dado lo negativa que es la conexión de B a C, un valor previamente calculado
que pasara por estos vertices no se pudiese recalcular dada la naturaleza miope
del algoritmo, lo cuál causaria que el resultado de el arbol que retorna un grafo con este
ciclo fuese incorrecto. Algoritmos de grafos dinámicos podrían hacer esta recalculación,
por lo que el algoritmo podría funcionar, en general, Dijkstra no sirve en grafos con
vértices negativos por esta razón.

%-------------------------------
%	CONTENIDO
%-------------------------------

\end{document}
