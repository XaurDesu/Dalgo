\documentclass[12pt]{exam}
\usepackage{amsthm}
\usepackage{libertine}
\usepackage[utf8]{inputenc}
\usepackage[margin=1in]{geometry}
\usepackage{amsmath,amssymb}
\usepackage{multicol}
\usepackage[shortlabels]{enumitem}
\usepackage{siunitx}
\usepackage{cancel}
\usepackage{graphicx}
\usepackage{pgfplots}
\usepackage{listings}
\usepackage{tikz}

\usepackage[T1]{fontenc}
\usepackage{lmodern}

\usepackage{listings}

\lstnewenvironment{DoxyVerb}{%
    \lstset{%
        basicstyle={\fontencoding{OT1}\fontfamily{cmr}\fontseries{b}\fontshape{n}\fontsize{10pt}{10}\selectfont},
        breaklines,
        language=C,
    }%
}{%
}

\pgfplotsset{width=10cm,compat=1.9}
\usepgfplotslibrary{external}
\tikzexternalize

\newcommand{\class}{Diseño y Análisis de Algoritmos} % This is the name of the course 
\newcommand{\examnum}{Tarea 1} % This is the name of the assignment
\newcommand{\examdate}{16/02/2023} % This is the due date
\newcommand{\timelimit}{}





\begin{document}
\pagestyle{plain}
\thispagestyle{empty}

\noindent
\begin{tabular*}{\textwidth}{l @{\extracolsep{\fill}} r @{\extracolsep{6pt}} l}
\textbf{\class} & \textbf{Name:} & \textit{Jaime Andres Torres Bermejo}\\ %Your name here instead, obviously 
\textbf{\examnum} &&\\
\textbf{\examdate} &&\\
\end{tabular*}\\
\rule[2ex]{\textwidth}{2pt}
% ---

\section*{Parte 1: Algoritmos a Analizar.}
Para los siguientes problemas:
\begin{enumerate}
    \item Proponga un algoritmo que solucione el problema. (Python)
    \item roponga una tabla que enumere las operaciones que se puede suponer que se ejecutan
    en tiempo constante
    \item Derive la función de costo, T(n), para el algoritmo que propuso, basado en las operaciones
    constantes del punto anterior.
    \item En caso de que la función sea una ecuación de recurrencia, resolver la ecuación.
    \item Determinar el orden de complejidad del algoritmo.
\end{enumerate}
\subsection*{Dada una matriz cuadrada de números enteros sumar los elementos de la
diagonal principal.}

\subsubsection*{1. Código Fuente.}
    \begin{verbatim}
    def matrix_sum(nums):
    i = len(nums)
    j = 0
    ret = 0
    while j < i:
        ret += nums[j][j]
        j += 1
    return ret
    \end{verbatim}

\subsubsection*{2. Tabla de costos}
\begin{center}
    \begin{tabular}{||c c||}
        $C_1$ & Asignación de Variables \\
        $C_2$ & Consultar 'len()' \\
        $C_3$ & Sumar variable\\
        $C_4$ & Comparación\\
    \end{tabular}
\end{center}

\subsubsection*{3. Derivar la función de costo para el algoritmo propuesto.}
$ T(n) = 3C_1 + C_2 + T(n(C_4+2C_1))$
\subsubsection*{4. esta NO es una ecuación de recurrencia.}
\subsubsection*{5. Determinar el orden de complejidad del algoritmo.}
$$ 3C_1 + C_2 + T(n(C_4+2C_1))$$
$$< \forall c \exists \mathbb{N} | cf = O(f) >$$
$$ T(n) = O(n) $$ 

El algoritmo es de complejidad O(n)

\subsection*{Dada una matriz de números enteros y un numero entero, se debe devolver
True si el numero se encuentra en la matriz y False en otro caso.}

\subsubsection*{1. Código Fuente.}

\begin{verbatim}
    def matrix_search(nums, target):
        for i in range(len(nums)):
            for j in range(len(nums[0])):
                if nums[i][j] == target:
                    return True
        return False
\end{verbatim}


\subsubsection*{2. Tabla de costos}
\begin{center}
    \begin{tabular}{||c c||}
        $C_1$ & Consultar 'len()' \\
        $C_2$ & Comparación\\
    \end{tabular}
\end{center}

\subsubsection*{3. Derivar la función de costo para el algoritmo propuesto.}
$$ T(n) = T(n(C_1)) + T(m(C_1+C_2)) $$
\subsubsection*{4. esta NO es una ecuación de recurrencia.}
\subsubsection*{5. Determinar el orden de complejidad del algoritmo.}

$$ T(n(C_1)) + T(m(C_1+C_2)) $$
$$< \forall c \exists \mathbb{N} | cf = O(f) >$$
$$ T(n) = O(n) + O(m) $$
$$ < Factorizacion > $$ 
$$ T(n) = O(n+m) $$

El algoritmo es de complejidad O(n+m), con n y m siendo ancho y largo de la
matriz.

\subsection*{Dada un arreglo de números enteros y un numero entero, se debe devolver
True si el numero se encuentra en el arreglo y False en otro caso. Solución recursiva}
\subsubsection*{1. Código Fuente.}
\begin{verbatim}
    def recursive_search(nums,target):
    if nums[0] == target:
        return True
    elif len(nums) == 1:
        return False
    recursive_search(nums[1:], target)
\end{verbatim}

\section*{Parte 2: Soluciones de las siguientes ecuaciones de recurrencia.}
determinar el orden de complejidad, para las ecuaciones no lineales debe:
\begin{enumerate}
    \item Proponer una cota usando arboles de recurrencia.
    \item Demostrar la cota propuesta usando el método de sustitución.
    \item De ser posible compruebe su respuesta usando el método maestro
\end{enumerate}
\subsection*{I.}
$ 2T(n) - T(n-1), T(0)=1, T(1)=5 $
\subsection*{Arbol de Recurrencia}
\begin{verbatim}
    2T(n) - T(n-1)

    .
    `-- T(0) = 1
        |-- T(1) = 5
        `-- T(2) = 3
            |-- T(2) = 3 
            `-- T(3) = 4
    --> O(n)
\end{verbatim}

\section*{Problema de Strassen}

El profesor Marco desea desarrollar un algoritmo de multiplicación de matrices
que sea asintóticamente más rápido que el algoritmo de Strassen. Su algoritmo utilizará el
método de dividir y conquistar, dividiendo cada matriz en trozos de tamaño n/4 x n/4, y los
pasos de dividir y combinar juntos tardarán O(n2). Necesita determinar cuántos subproblemas
tiene que crear su algoritmo para vencer al algoritmo de Strassen. Si su algoritmo crea
un subproblema, entonces la recurrencia para el tiempo de ejecución T(n) se convierte en
T(n) = aT (n/4) + O(n2) ¿Cuál es el mayor valor entero de a para el que el algoritmo del
profesor Marco sería asintóticamente más rápido que el algoritmo de Strassen?

\subsection*{Solución}
\subsubsection*{Algoritmo de Strassen.}
La complejidad del algoritmo de Strassen esta definida como $ \log _{10} 3 $ \cite[CENK2017484]{keylist}


\end{document}

@article{CENK2017484,
title = {On the arithmetic complexity of Strassen-like matrix multiplications},
journal = {Journal of Symbolic Computation},
volume = {80},
pages = {484-501},
year = {2017},
issn = {0747-7171},
doi = {https://doi.org/10.1016/j.jsc.2016.07.004},
url = {https://www.sciencedirect.com/science/article/pii/S0747717116300359},
author = {Murat Cenk and M. Anwar Hasan},
keywords = {Fast matrix multiplication, Strassen-like matrix multiplication, Computational complexity, Cryptographic computations, Computer algebra},
abstract = {The Strassen algorithm for multiplying 2×2 matrices requires seven multiplications and 18 additions. The recursive use of this algorithm for matrices of dimension n yields a total arithmetic complexity of (7n2.81−6n2) for n=2k. Winograd showed that using seven multiplications for this kind of matrix multiplication is optimal. Therefore, any algorithm for multiplying 2×2 matrices with seven multiplications is called a Strassen-like algorithm. Winograd also discovered an additively optimal Strassen-like algorithm with 15 additions. This algorithm is called the Winograd's variant, whose arithmetic complexity is (6n2.81−5n2) for n=2k and (3.73n2.81−5n2) for n=8⋅2k, which is the best-known bound for Strassen-like multiplications. This paper proposes a method that reduces the complexity of Winograd's variant to (5n2.81+0.5n2.59+2n2.32−6.5n2) for n=2k. It is also shown that the total arithmetic complexity can be improved to (3.55n2.81+0.148n2.59+1.02n2.32−6.5n2) for n=8⋅2k, which, to the best of our knowledge, improves the best-known bound for a Strassen-like matrix multiplication algorithm.}
}