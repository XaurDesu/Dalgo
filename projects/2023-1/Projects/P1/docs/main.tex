\documentclass[a4paper]{article} 
\input{head}
\begin{document}

%-------------------------------
%	TITLE SECTION
%-------------------------------

\fancyhead[C]{}
\hrule \medskip % Upper rule
\begin{minipage}{0.295\textwidth} 
\raggedright
\footnotesize
Jaime Andres Torres Bermejo \hfill\\   
202014866\hfill\\
andrestbermejoj@gmail.com
\end{minipage}
\begin{minipage}{0.4\textwidth} 
\centering 
\large 
Proyecto 1\\ 
\normalsize 
Diseño y Análisis de Algoritmos\\ 
\end{minipage}
\begin{minipage}{0.295\textwidth} 
\raggedleft
\today\hfill\\
\end{minipage}
\medskip\hrule 
\bigskip

%-------------------------------
%	CONTENIDO
%-------------------------------
\section{Enunciado del Proyecto}

Uno de los principales retos de la reforma a la salud 2023 propuesta por el gobierno actual son los
denominados Centros de Atención Primaria (CAD). A cada CAD en el país serán asignados los
beneficiarios y sus familias según cercanía y disponibilidad. Para que las finanzas de estos centros
no se vean afectadas por una mala distribución de beneficiarios y para garantizar el mejor servicio,
es necesario diseñar una estrategia de asignación equitativa de beneficiarios.

\paragraph{Problema}
Suponga que para un atender la demanda de salud en un determinado municipio de Colombia se
habilitaran \textit{k} diferentes CADs a los cuales se deberán asignar \textit{m} familias diferentes. La familia i-
esima cuenta con \textit{$f_i$} miembros diferentes. Se le encarga la tarea de identificar si existe una manera
en que todos los CAD puedan quedar con la misma cantidad de beneficiarios, con la restricción que
TODOS los miembros de un mismo grupo familiar sean asignados a un mismo CAD.


%------------------------------------------------

\section{First Exercise}
\blindtext
\subsection{First Subtask}
Some equations
\begin{align*}
y &=  \sum\limits_{i,k} m_i \cdot f^k \\
x &=  
\underset{11}{\underbrace{3 + 8}} + 5 + 7
\end{align*}

\subsection{Second Subtask}
\blindtext

\bigskip

%------------------------------------------------

%------------------------------------------------

\end{document}
