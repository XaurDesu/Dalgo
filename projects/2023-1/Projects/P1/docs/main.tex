\documentclass[a4paper]{article} 
\addtolength{\hoffset}{-2.25cm}
\addtolength{\textwidth}{4.5cm}
\addtolength{\voffset}{-3.25cm}
\addtolength{\textheight}{5cm}
\setlength{\parskip}{0pt}
\setlength{\parindent}{0in}

%----------------------------------------------------------------------------------------
%	PACKAGES AND OTHER DOCUMENT CONFIGURATIONS
%----------------------------------------------------------------------------------------

\usepackage{blindtext} % Package to generate dummy text
\usepackage{charter} % Use the Charter font
\usepackage[utf8]{inputenc} % Use UTF-8 encoding
\usepackage{microtype} % Slightly tweak font spacing for aesthetics
\usepackage[english, ngerman]{babel} % Language hyphenation and typographical rules
\usepackage{amsthm, amsmath, amssymb} % Mathematical typesetting
\usepackage{float} % Improved interface for floating objects
\usepackage[final, colorlinks = true, 
            linkcolor = black, 
            citecolor = black]{hyperref} % For hyperlinks in the PDF
\usepackage{graphicx, multicol} % Enhanced support for graphics
\usepackage{xcolor} % Driver-independent color extensions
\usepackage{marvosym, wasysym} % More symbols
\usepackage{rotating} % Rotation tools
\usepackage{censor} % Facilities for controlling restricted text
\usepackage{listings, style/lstlisting} % Environment for non-formatted code, !uses style file!
\usepackage{pseudocode} % Environment for specifying algorithms in a natural way
\usepackage{style/avm} % Environment for f-structures, !uses style file!
\usepackage{booktabs} % Enhances quality of tables
\usepackage{tikz-qtree} % Easy tree drawing tool
\tikzset{every tree node/.style={align=center,anchor=north},
         level distance=2cm} % Configuration for q-trees
\usepackage{style/btree} % Configuration for b-trees and b+-trees, !uses style file!
\usepackage[backend=biber,style=numeric,
            sorting=nyt]{biblatex} % Complete reimplementation of bibliographic facilities
\addbibresource{ecl.bib}
\usepackage{csquotes} % Context sensitive quotation facilities
\usepackage[yyyymmdd]{datetime} % Uses YEAR-MONTH-DAY format for dates
\renewcommand{\dateseparator}{-} % Sets dateseparator to '-'
\usepackage{fancyhdr} % Headers and footers
\pagestyle{fancy} % All pages have headers and footers
\fancyhead{}\renewcommand{\headrulewidth}{0pt} % Blank out the default header
\fancyfoot[L]{} % Custom footer text
\fancyfoot[C]{} % Custom footer text
\fancyfoot[R]{\thepage} % Custom footer text
\newcommand{\note}[1]{\marginpar{\scriptsize \textcolor{red}{#1}}} % Enables comments in red on margin

%----------------------------------------------------------------------------------------

\begin{document}

%-------------------------------
%	TITLE SECTION
%-------------------------------

\fancyhead[C]{}
\hrule \medskip % Upper rule
\begin{minipage}{0.295\textwidth} 
\raggedright
\footnotesize
Jaime Andres Torres Bermejo \hfill\\   
202014866\hfill\\
andrestbermejoj@gmail.com
\end{minipage}
\begin{minipage}{0.4\textwidth} 
\centering 
\large 
Proyecto 1\\ 
\normalsize 
Diseño y Análisis de Algoritmos\\ 
\end{minipage}
\begin{minipage}{0.295\textwidth} 
\raggedleft
\today\hfill\\
\end{minipage}
\medskip\hrule 
\bigskip

%-------------------------------
%	CONTENIDO
%-------------------------------
\section{Enunciado del Proyecto}

Uno de los principales retos de la reforma a la salud 2023 propuesta por el gobierno actual son los
denominados Centros de Atención Primaria (CAD). A cada CAD en el país serán asignados los
beneficiarios y sus familias según cercanía y disponibilidad. Para que las finanzas de estos centros
no se vean afectadas por una mala distribución de beneficiarios y para garantizar el mejor servicio,
es necesario diseñar una estrategia de asignación equitativa de beneficiarios.

\paragraph{Problema}
Suponga que para un atender la demanda de salud en un determinado municipio de Colombia se
habilitaran \textit{k} diferentes CADs a los cuales se deberán asignar \textit{m} familias diferentes. La familia i-
esima cuenta con \textit{$f_i$} miembros diferentes. Se le encarga la tarea de identificar si existe una manera
en que todos los CAD puedan quedar con la misma cantidad de beneficiarios, con la restricción que
TODOS los miembros de un mismo grupo familiar sean asignados a un mismo CAD.
\section{Especificaciones del proyecto}

\subsection{Entorno de Desarrollo y entorno de prueba.}

La solución propuesta fue probada en el siguiente entorno de Desarrollo, debería funcionar en
computadores que compartan las dependencias principales, como Python o un sistema operativo basado
en Unix, así no comparta necesariamente el hardware. No se proporciona un entorno de virtualización
por lo que recomiendo intentar correrlo en un entorno relativamente similar.

\begin{itemize}
    \item \textbf{Sistema Operativo: } Ubuntu 22.10 x86\_64, Minimal Install.
    \item \textbf{Kernel: }Linux 5.19.0-38-generic
    \item \textbf{Lenguaje de Programación: }Python 3.10.7 x86\_64, Compilado desde apt como un .deb, incluye python3-pip
    \item \textbf{Librerías Utilizadas}
    \begin{itemize}
        \item os
    \end{itemize}
    \item \textbf{Procesador: }AMD Ryzen 5 3550H with Radeon Vega Mobile Gfx (8) @ 2.100GHz 
    \item \textbf{IDE: } Visual Studio Code 1.76.2, instalado desde un paquete snapd. Incluye extensiones
    de Git y Python.
\end{itemize}

\subsection{Estructura del Input}
A primera vista, este pareciese ser un problema bastante extraño, si fuesemos a tomarlo por la lógica
del propio problema sin necesariamente cuestionarla. Sin embargo, esta estructura puede verse de forma mucho
más clara al ver la forma como un input esta estructurado, y a partir de esto, comprobar la
utilidad de cada parte del input. Entonces, tomemos como ejemplo el segundo test proveido por el
enunciado:
\begin{verbatim}
    3 
    3 5 8 7 3 4 11
    2 3 4 7 9 
    2 4 5 7 8 4
\end{verbatim}

%------------------------------------------------

\begin{align*}
y &=  \sum\limits_{i,k} m_i \cdot f^k \\
x &=  
\underset{11}{\underbrace{3 + 8}} + 5 + 7
\end{align*}

%------------------------------------------------

%------------------------------------------------

\end{document}
