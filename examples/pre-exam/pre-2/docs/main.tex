\documentclass[a4paper]{article} 
\input{head}
\begin{document}

%-------------------------------
%	TITLE SECTION
%-------------------------------

\fancyhead[C]{}
\hrule \medskip % Upper rule
\begin{minipage}{0.295\textwidth} 
\raggedright
\footnotesize
Jaime Andres Torres Bermejo \hfill\\   
202014866\hfill\\
andrestbermejoj@gmail.com
\end{minipage}
\begin{minipage}{0.4\textwidth} 
\centering 
\large 
Pre parcial 2\\ 
\normalsize 
Diseño y Análisis de Algoritmos\\ 
\end{minipage}
\begin{minipage}{0.295\textwidth} 
\raggedleft
\today\hfill\\
\end{minipage}
\medskip\hrule 
\bigskip

%-------------------------------
%	CONTENIDO
%-------------------------------
\section{Implemente todos los algoritmos vistos en clase y estudie su complejidad temporal.}
\subsection{DFS}
\begin{verbatim}
def dfs(Adj, s, visited): 
    visited[s] = True
    for n in Adj[s]:
        if not visited[n]:
        dfs(Adj,n,visited)
\end{verbatim}

\begin{itemize}
    \item Uso: Recorre todos los nodos de un grafo
    \item Complejidad:
    
\end{itemize}

\subsection{BFS}
\begin{verbatim}
def bfs(Adj, s): 
    visited = [False for v in Adj] 
    queue = [s]
    while 0 < len(queue): 
      current_node = queue.pop(0)
      if not visited[current_node]:
       # Usually here you do something 
       # with the current node
       visited[current_node] = True
       for n in Adj[current_node]:
         if not visited[n]:
           queue.append(n)
\end{verbatim}

\begin{itemize}
    \item Uso: Recorre todos los nodos de un grafo
    \item Complejidad:
    
\end{itemize}

\section{Explique cual es la complejidad temporal de los algoritmos vistos en clase: BFS,
Dijkstra, Bellman-Ford y Floyd-Warshall.}

%-------------------------------

\end{document}
