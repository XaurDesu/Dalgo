\documentclass[a4paper]{article} 
\addtolength{\hoffset}{-2.25cm}
\addtolength{\textwidth}{4.5cm}
\addtolength{\voffset}{-3.25cm}
\addtolength{\textheight}{5cm}
\setlength{\parskip}{0pt}
\setlength{\parindent}{0in}

%----------------------------------------------------------------------------------------
%	PACKAGES AND OTHER DOCUMENT CONFIGURATIONS
%----------------------------------------------------------------------------------------

\usepackage{blindtext} % Package to generate dummy text
\usepackage{charter} % Use the Charter font
\usepackage[utf8]{inputenc} % Use UTF-8 encoding
\usepackage{microtype} % Slightly tweak font spacing for aesthetics
\usepackage[english, ngerman]{babel} % Language hyphenation and typographical rules
\usepackage{amsthm, amsmath, amssymb} % Mathematical typesetting
\usepackage{float} % Improved interface for floating objects
\usepackage[final, colorlinks = true, 
            linkcolor = black, 
            citecolor = black]{hyperref} % For hyperlinks in the PDF
\usepackage{graphicx, multicol} % Enhanced support for graphics
\usepackage{xcolor} % Driver-independent color extensions
\usepackage{marvosym, wasysym} % More symbols
\usepackage{rotating} % Rotation tools
\usepackage{censor} % Facilities for controlling restricted text
\usepackage{listings, style/lstlisting} % Environment for non-formatted code, !uses style file!
\usepackage{pseudocode} % Environment for specifying algorithms in a natural way
\usepackage{style/avm} % Environment for f-structures, !uses style file!
\usepackage{booktabs} % Enhances quality of tables
\usepackage{tikz-qtree} % Easy tree drawing tool
\tikzset{every tree node/.style={align=center,anchor=north},
         level distance=2cm} % Configuration for q-trees
\usepackage{style/btree} % Configuration for b-trees and b+-trees, !uses style file!
\usepackage[backend=biber,style=numeric,
            sorting=nyt]{biblatex} % Complete reimplementation of bibliographic facilities
\addbibresource{ecl.bib}
\usepackage{csquotes} % Context sensitive quotation facilities
\usepackage[yyyymmdd]{datetime} % Uses YEAR-MONTH-DAY format for dates
\renewcommand{\dateseparator}{-} % Sets dateseparator to '-'
\usepackage{fancyhdr} % Headers and footers
\pagestyle{fancy} % All pages have headers and footers
\fancyhead{}\renewcommand{\headrulewidth}{0pt} % Blank out the default header
\fancyfoot[L]{} % Custom footer text
\fancyfoot[C]{} % Custom footer text
\fancyfoot[R]{\thepage} % Custom footer text
\newcommand{\note}[1]{\marginpar{\scriptsize \textcolor{red}{#1}}} % Enables comments in red on margin

%----------------------------------------------------------------------------------------

\begin{document}

%-------------------------------
%	TITLE SECTION
%-------------------------------

\fancyhead[C]{}
\hrule \medskip % Upper rule
\begin{minipage}{0.295\textwidth} 
\raggedright
\footnotesize
Jaime Andres Torres Bermejo \hfill\\   
202014866\hfill\\
andrestbermejoj@gmail.com
\end{minipage}
\begin{minipage}{0.4\textwidth} 
\centering 
\large 
Pre parcial 2\\ 
\normalsize 
Diseño y Análisis de Algoritmos\\ 
\end{minipage}
\begin{minipage}{0.295\textwidth} 
\raggedleft
\today\hfill\\
\end{minipage}
\medskip\hrule 
\bigskip

%-------------------------------
%	CONTENIDO
%-------------------------------
\section{Implemente todos los algoritmos vistos en clase y estudie su complejidad temporal.}
\subsection{DFS}
\begin{verbatim}
def dfs(Adj, s, visited): 
    visited[s] = True
    for n in Adj[s]:
        if not visited[n]:
        dfs(Adj,n,visited)
\end{verbatim}

\begin{itemize}
    \item Uso: Recorre todos los nodos de un grafo
    \item Complejidad: O(V+E)
    
\end{itemize}

\subsection{BFS}
\begin{verbatim}
def bfs(Adj, s): 
    visited = [False for v in Adj] 
    queue = [s]
    while 0 < len(queue): 
      current_node = queue.pop(0)
      if not visited[current_node]:
       # Usually here you do something 
       # with the current node
       visited[current_node] = True
       for n in Adj[current_node]:
         if not visited[n]:
           queue.append(n)
\end{verbatim}

\begin{itemize}
    \item Uso: Recorre todos los nodos de un grafo
    \item Complejidad: O(V+E)
    
\end{itemize}


\subsection{Dijkstra}
\begin{verbatim}
    def dijkstra(graph, start, end):
    """Return the cost of the shortest path between vertices start and end.
    >>> dijkstra(G, "E", "C")
    6
    >>> dijkstra(G2, "E", "F")
    3
    >>> dijkstra(G3, "E", "F")
    3
    """

    heap = [(0, start)]  # cost from start node,end node
    visited = set()
    while heap:
        (cost, u) = heapq.heappop(heap)
        if u in visited:
            continue
        visited.add(u)
        if u == end:
            return cost
        for v, c in graph[u]:
            if v in visited:
                continue
            next_item = cost + c
            heapq.heappush(heap, (next_item, v))
    return -1
\end{verbatim}

\begin{itemize}
    \item Uso: Recorre todos los nodos de un grafo
    \item Complejidad: O(V^2)
    
\end{itemize}


\section{Explique cual es la complejidad temporal de los algoritmos vistos en clase: BFS,
Dijkstra, Bellman-Ford y Floyd-Warshall.}

%-------------------------------

\end{document}
