\documentclass[12pt,a4paper,teal]{bbe}
\usepackage{blindtext}
\begin{document}
	\chapter*{Recursion}
	Recursion in this context is defined as a finite expression that implies
	succesion.
	\section{Recursive Equation}
	A recursive equation describes the value of the element of a succesion 
	parting from previous elements of that specific succesion. These can be defined
	in either a linear or non-linear way, depending on the problem to be solved by
	this recursive algorithm.
	\section{Linear Equations}
	\subsection{Homogeneous Linear Recursive Relation}
	A Homogeneous linear recursive relation is a relation that can be defined
	in mathematical terms as a relation that could be expressed by the formula:

	$$ \sum_{i = 0}^{k} a_i f(n-1) = 0 $$

	and that includes a characteristic polinomial that can be expressed as:
	$$\sum_{i = 0}^{k} a_i \lambda^(n-1)$$

	\subsection{Non-linear Recursive Relation}
	A non linear recursive relation is a relation we can explain as defined by the formula
	$$ \sum_{i = 0}^{k} a_i f(n-1) = \mathbb{R} $$

	where the solution is different from zero.


	\subsection{Examples}

	We suppouse a $T(n)$ of the form:
	$$ T(n) = \sum_{i=0}^{k} aT(n-i)=g(n) $$
	we would have the solution:
	$$T(n) = h(n)+P(n)$$
	Where:

	\begin{itemize}
		\item $g(n)=c \rightarrow P(n)=C_0$
		\item $ g(n) = Cn + c' \rightarrow P(n) = C_0n+C_1 $
	\end{itemize}

	\subsubsection{Example 1}
	For the case in which T(n) is
	$$ T(n) = \begin{cases} 
	2\\
	3n+3T(n-1)
	\end{cases}  $$
	find:
	\begin{itemize}
		\item h(n)
		\item p(n)
	\end{itemize}
	\paragraph*{Solution}
	to find h(n), first we define the first characteristic polynomial as:
	$$
	\lambda -3=0
	$$
	$$
	\lambda = 3
	$$

	And the second as:
	$$ 2 =  C_0^1 3^1$$
	$$ \frac{2}{3} = C_0^1 $$


	And therefore, we conclude
	h(n) = $C_0^1 3^n$ and h(n) = $2 * 3^{n-1}$
	
	\section{Non-Linear Equations}

	\subsection{Recurrent non-linear equations.}
	A recurrent non-linear equation is,

	\subsubsection{Recursion Tree}
	this is a graphical method for solving non-linear equations. The 
	solution of this method is defined by the recurrent decomposition of
	a specific mathematical expression. Arriving to 
	
	\subsubsection{Substitution Method.}

	\subsubsection{Master Method.}
\end{document}

